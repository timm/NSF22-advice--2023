
 
  RQ4 (year2) : Are
decision making styles differenrt for different catetofies of people? trigger of different cues,es different inforamtion colelction startegies?

=> review our logs to see if their are style differenetce

==> predicting this willr evise must in our tools. at least the itnerface. maybe ven inspire differeeces in hwo to algoroithms work.

XXX gender and race are definte  targets we want to study. Other groupsing will studied, if we can cann ecess enough stahefhilders in those groups. and depending  (in this work or in future work)

 
  RQ5: (year3) Can \ITS{3} mitigate for bad advice?


For the purposes of this work, we say
advice is ``faulty'' if it is associated with a bug fixing change, defined as follows.
Suppose we are   exploring
model   for   several days (or longer); e.g.
\bi
\item A team must complete a complex task that requires extensive negotiation;
\item The team is using be reviewing (or updating) old decisions (for auditing or maintenance purposes)
\ei
In this context, there will be two kinds of changes (changes to the models, changes to the advice):
\bi
\item
Changes due to proposed enhancements 
\item
Other changes,  which we would call ``bug fixes''.
CROWDED is 
\underline{\bf faulty} if,      subsequently, it needs   bug fixes.
\ei

We weill comapre three treatments 
 


\underline{\em RX5a Traditional manual
repaor  methods:}~\cite{Easterbrook08},
it is standard to   take advice from two people it is standard to   take advice from two people 
and, if they cannot agree, then ask a third to resolve the dispute.

\underline{\em RX5b Adapting qualitative methods: }
 In this research, we can at least partially automate that ``gang of three'' approach since our $x$ spaces will be annotated with advice from multiple stakeholders.  That is, once faulty advice is found, we can replay the flight of our particles, deleting the influence of the faulty advice.
%then synthesize alternative advice from the landscape around the particle. \textcolor{green}{Not sure if it will work for the tool? Should we remove it?. see comment
 


\underline{\em RX5c Adapting quantitative methods:}
co-PI Menzies
has been experimenting with Yu et al.~\cite{yu2019improving}  as follows.  Given a model  executed over an extended period of time,   inputs could be presented to later and older models.
If (e.g.) Wednesday's model
gives different output to Monday's model then that raises a red flag on all the advice from Monday to
Wednesday. As above, we could then replay the flight of the particles, synthesize alternative advice from the landscape around the particle. 
Using delta debugging techniques~\cite{Zeller99}, the difference in the model generated with and without the red flag advice would be something that could be presented to stakeholders for their review. 
 

% \item RQ4: Is CROWDED a training tool? i.e. can we use prior    CROWDED to offer guidance (training, assistance) to newcomers ? CROWDED3 will evaluate.
\item RQ6: (year3)Is \ITS{4} a tool for social justice? I.e.can it be used to change designs in order to empower the disempowered?  Or will CROWDED need the bias mitigation tools. CROWDED4 will evaluate.

XXX
