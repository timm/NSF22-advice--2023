
 
\begin{table}[!t]
\caption{ In M-of-N studies,  a few $M$ humans monitor a large number of $N$ drones. }\label{mns}
{\small
\begin{tabular}{|p{.98\linewidth}|}\hline
\rowcolor{blue!10}
In an  {\em  m-of-n} study,  teams of $M$ humans will play a simulation game in which they must 
 monitor a fleet of   $N$ drones,   watching for off-nominal
situations, intervening when necessary.  \\
Just to give a flavor of the off-nominal problems our humans might be asked
to manage, here a few examples:
\bi
\item An emergency  team laying down a back burn must have a workable escape route.
Hence, if that team is working at one end of road and,   several miles away, a brush fire is encroaching   the escape route, then we must intervene to order the crews to down tools and rush their escape.
\item 
Drones in urban areas can fly up to 400' while Cessnas, monitoring those drones,
can fly down to 500'. In the case of approaching high winds, we must intervene to order the Cessnas to fly higher. \ei\\
\rowcolor{blue!10}
Note the monitoring challenge of the above two examples. Firstly, to see the problem, humans have to synthesize connections between
events occurring in remote parts of the simulation game.
Secondly, each drone in isolation could be working perfectly.
But when sets of drones co-ordinate with humans in a shared
space, some problems (like those above) become emergent.\\\hline
\end{tabular}}
\end{table}
