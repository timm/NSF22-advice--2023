Panel Summary
Panel Summary

1. PROPOSAL SUMMARY:


The proposal is in the area of requirements engineering for software. The problem the proposal targets is that developers sometimes face complex decision-making about how to allocate resources in building software. For example, choosing between two important features when there are only resources for one, or choosing whether to prioritize fixing a bug versus implementing a new feature. In a large system, it can become overwhelming to choose an optimal path. A trend in research is to provide AI-based support tools. This proposal builds on that trend by seeking to unite human- and machine-driven optimization for requirements engineering.

The proposal divides the novel effort into four thrusts: 1) integrating team-based input into AI-based requirements engineering, 2) integrating cloud-based input, 3) verification of output suggestions from the proposed work, and 4) bias mitigation in AI-guided search.

2. INTELLECTUAL MERIT:



The strengths of the intellectual merit include:
+ The proposal includes unique, out-of-the-box ideas to solve very complex requirements engineering problems.
+ The idea of combining team- and cloud-based input seems like a viable way forward given the state of current technology.
+ The PIs have a strong history of success in related areas.

The weaknesses of the intellectual merit include:
- The proposal lacks detail at several crucial points, especially in Thrusts 3 and 4. The proposal does not provide enough detail to understand the key factors that will lead to the project’s success.
- Some areas of the proposal are very broad, spreading the proposed work too thinly. Thrust 4, for example, considers several areas of bias without sufficient focus.
- The proposal is not clear about the scenarios for which data will be collected and in which the work will be evaluated.
- The proposal is not clear about the interface with which the humans will interact. The core of the proposed work involves human involvement with machine decision-making. The interface the humans use could have an impact on the success of the project. The proposal does not address this risk.
- The proposal uses goal models out of context and for a different purpose without justification.



3. BROADER IMPACTS:



The strengths of the broader impacts include:
+ If successful, the project would lead to improved use of AI-based models by integrating those models with human intuition.
+ The PIs could build on previous success in improving the participation of women in computing.
+ The project includes integration of the work into one of the PIs' university’s admission system. If the project were to be successful, the work on the admission system could help mitigate bias at that university.

The weakness of the broader impacts include:
- The proposal is not clear about what will be demonstrated with respect to the admissions system.



4. SUGGESTED IMPROVEMENTS:



The panel felt that the strongest area of the proposal is in Thrust 1, related to the team-based feedback for AI-driven requirements engineering. The proposal would be strengthened if it focused on this work instead of spreading itself thinly over the several other areas. A strong suggestion is to better articulate the scenarios around which experiments will be conducted and around which the PIs envision the work being used in practice.



5. CONCLUSIONS:


The proposed work includes several unique strategies for tackling challenging problems in requirements engineering. However, the work is spread too thinly in too many directions. The proposal lacks sufficient details in key areas. More focus and better articulation of the usage scenarios could strengthen the proposal.



Panel recommendation: Low Competitive