
The PIs will take responsibility and provide oversight for the management of data associated with the proposed research. The students associated with this project will be briefed on and required to adhere to the data management plan. In addition, PIs and students associated with this project will receive appropriate IRB training and certification that includes information and best practices related to the ethical management of human subjects’ data. This will allow the data generated from the project to be shared across institutions to the extent allowed by the IRB board at NCSU in accordance with NSF policies.

\subsection{Data generated and access}
As per our universities’ IRB policies, the data throughout the research will be collected only from participants who agree. All materials will be made anonymous and stored locally under password protection, so as to be available only to the research team and managed under the restrictions imposed by the IRB approval obtained by the PIs’ institution. In particular, data identifying participants will be erased completely as soon as anonymization is complete. Additional metadata will include coding schemes and memos, experiment protocols, design recommendations and derivative design ideas, and scenarios of our proposed tool (\textit{ADVICE}). However, aggregations (descriptive and inferential statistics and selected, anonymized short excerpts of quotes, logs, or behavior sequences) will be disseminated through scholarly publication, as will details of the study designs, methodologies, and procedures used to collect these data. 
%\subsubsection{Data from Labs and Case Studies}

\textbf{Data gathered in the user studies.} This data will include qualitative (observational field notes, recorded participants’ verbalizations and behaviors) and quantitative data (questionnaires, log data, digital audio/video recordings of studies with human subjects, screen captures of user experiment sessions, transcriptions of interview results, and programs) representing the performance, outcomes, and attitudes of each participant. The videos of participants will be transcribed and labeled with unique identifiers to anonymize them. 

\textbf{ADVICE Usage Data.} 
Participant’s assessment results will be available only to the participants and the research teams and managed under the restrictions imposed by the IRB approval obtained by the PI’s institutions but will be disseminated in the aggregate through scholarly publication. Participants will have the option of publicly displaying their progress in the online community. 

\subsection{Dissemination and sharing of results}
Key excerpts of anonymized primary data will be routinely shared in the context of publications of this research. Final coding schemes, design recommendations, and design prototypes that serve to synthesize across and characterize our holistic understanding of the primary data will be shared as well, in the context of publications of this work. All publications resulting from this research will include relevant methodological information necessary to understand how the data was collected and analyzed. In the event that key data has not been published in formal venues within 3 years of the termination of this fellowship, all unpublished manuscripts will be made publically available as tech notes or working papers. Human subject’s data that include privileged and confidential information and raw data will be kept secure in accordance to approved IRB protocols and best practices for protection of human subjects. Any human subject’s data that is publicly released or shared will be suitably anonymized and aggregated to maintain confidentiality and protect the privacy of subjects.

We will promptly prepare and submit for publication, with authorship that accurately reflects the contributions of those involved, all significant findings from work performed in this project. To promote widespread use and dissemination of our research results, our first choices for publishing this research will be at venues that allow authors to make digital copies of their publications freely available via their personal websites (e.g., ACM conferences). To the extent that this is possible, then, all publications (including tech notes and working papers) resulting from this research will be made publically available on the project website as pdf files. 

\subsection{Software}
A dedicated website will be created for the project and the entire project related annotated data, ADVICE software access, publications, and relevant documents will made accessible through the website. ADVICE  will use standard software development practices and will be made open source under \textcolor{black}{the GPL} license. The ADVICE prototype will be distributed, when allowable by PIs' tech transfer office, with source code, and through project web site for at least three years, so that it is available to and modifiable by anyone in the world. The ADVICE tool source may also be hosted on an open source project site such as GitHub, which will allow our team to coordinate our development efforts as well as allow the general public to download the prototype as they see fit. 

This research will also produce public source code; for example  structured data sets; predictive models; predictive model's intermediate tuning cache; records of results of applying the code to the data;  and work-processing files (the reports of our results). 
All data will be stored in different formats, appropriate for the type of data being stored. For example, appropriate data formats for papers include word files, latex files, and PDF files. As another example, appropriate data formats for repository data include Attribute Relation File Format (.arff) or Comma-Separated Values (.csv). Similarly Hierarchical Data Format (.h5) or Pickle(.pkl) for storing generated models. As to everything else,
where possible, open source data and software will be shared publicly through a repository (with an exception for qualitative survey results which may be presented as a survey summary after applying de-identification of the responses from individuals). 

Repository data will be freely available to the world and \textcolor{black}{licensed under Creative Commons Attribution 4.0. International license (https://creativecommons.org/ licenses/by/4.0/).} This data will be made freely available in our repository, accessible through the internet.  All published and open source data may be re-used, re-distributed, and derived as long as the original data is not misrepresented, and the materials do not violate the acceptable use of the IP holders. Others may change or alter the open source data for personal purposes provided that these changes are made clear in any publications, re- distributions, or derivations. 

Repository data will be hosted on the Zenodo repository (hosted at the Large Hadron Collider in Switzerland). Papers will be published at major conferences and journals in SE data analytics.
Project results will be archived at the University on its web server, managed and supported by the college’s IT team. Individual human-subjects data will be archived on a password-protected system or in locked cabinets accessible to only the PI and the research team. Raw data will be archived for a limited period of time of no more than 3 years from the completion of the project as required by approved human-subjects protocols; after this period of time has expired, the data will be destroyed to protect the privacy of individuals involved.


\subsection{Data Sharing with NASA}
We are slated to conduct studies at NASA, following the approval from both the Institutional Review Board (IRB) at NCSU and the NASA IRB. Only anonymized data will be shared with our NASA collaborator.