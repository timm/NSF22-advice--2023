
\begin{table}[!t]
\caption{Notes on our human-intensive studies.}\label{studies}
{\small
\begin{tabular}{|p{.98\linewidth}|}\hline 
\rowcolor{blue!10}
User studies will be conducted after the approval for
 NC State Investigator Review
Board (IRB). \\ 

We will match   user studies to the participant skills.
  Simple user studies (e.g. Amazon drones delivering packages to households, during
a thunderstorm) will be used for our work with NC State students.
On the other hand, 
the fmdtools models of \S\ref{egmodel} will be used in association with subject matter experts from the SMART-STEReO team. 
\\\rowcolor{blue!10}
Initially, we will work with teams of size $N=2$, then move to larger teams for years two and three.\\
Initially, we will run   simple user studies (e.g.a fleet of drones  delivering Amazon packages during a thunderstorm)
then  move to move complex studies. Finally, in year 3 we will explore  
{\bf case studies} based on the emergency response teams modeled in the fmdtool models generated
by the SMART-STEReO team.
\\\rowcolor{blue!10}
For our  initial small \underline{\em lab studies}
we will recruit university students. \\\rowcolor{blue!10}
For our \underline{\em case studies}, we will recruit  real-world subject matter
experts from the SMART-STERe team.
\\
 All our lab studies and case studies will explore models at various levels of ``stress''.
Stressing factors could be ``achieve more goals with fewer drones''; ``achieve more goals in less time'';
``achieve more goals with fewer team members'';
``achieve more goals in the presence of an increasing number of non-nominal scenariois'';
``achieve more goals when the time-to=failure of each drone becomes increasingly variable''.   \\\rowcolor{blue!10}
For data collection in these lab studies and case studies, we will:
\bi 
\item Initially use a
  think-aloud protocol \cite{lewisusing, Seaman1999} 
  where participants will   vocalize their thoughts and feelings as they perform their tasks.
  \item Conduct retrospective interviews to gain insights into participant experiences and barriers encountered during decision making with and without {\IT}.
  \item At the end, we will measure the cognitive load of the humans and usability of {\IT}. \ei \\
As in our past qualitative analyses (e.g., \cite{Kuttal2020, Kuttal2021g}), we will triangulate (compare and attempt to refute) the results with interviews and surveys to understand their strengths. %Retrospective interviews will be conducted to gain insights into participants' experiences and barriers encountered during decision-making with and without {\IT}. 
We will collect (1) Pre-surveys with standard questionnaires to collect demographics and familiarity with the domain of the model. (2) Pre- and post-surveys with standard questionnaires will be used to evaluate self-efficacy. (3)  User Experience Questionnaire (UEQ)~\cite{rauschenberger2013efficient,schankin2022psychometric,schrepp2014applying} to evaluate the usability of  {\IT} interface. (4) Measure participant engagement with the {\IT} tool using the ISA engagement scale \cite{soane2012development}, which is based on the view that engagement comprises 'intellectual,' 'social,' and 'affective' dimensions. (5) Post-surveys will be used to analyze the cognitive load \cite{CognitiveLoad}(such as using NASA Task Load Index~\cite{hart2006nasa})  to measure and conduct a subjective mental workload (MWL) assessment. \\\hline
\end{tabular}}
\end{table}